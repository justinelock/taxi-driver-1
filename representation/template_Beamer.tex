%%%%%%%%%%%%%%%%%%%%%%%%%%%%%%%%%%%%%%%%%
% Beamer Presentation
% LaTeX Template
% Version 1.0 (10/11/12)
%
% This template has been downloaded from:
% http://www.LaTeXTemplates.com
%
% License:
% CC BY-NC-SA 3.0 (http://creativecommons.org/licenses/by-nc-sa/3.0/)
%
%%%%%%%%%%%%%%%%%%%%%%%%%%%%%%%%%%%%%%%%%

%----------------------------------------------------------------------------------------
%	PACKAGES AND THEMES
%----------------------------------------------------------------------------------------

\documentclass{beamer}

\mode<presentation> {
	
	% The Beamer class comes with a number of default slide themes
	% which change the colors and layouts of slides. Below this is a list
	% of all the themes, uncomment each in turn to see what they look like.
	
	%\usetheme{default}
	%\usetheme{AnnArbor}
	%\usetheme{Antibes}
	%\usetheme{Bergen}
	%\usetheme{Berkeley}
	%\usetheme{Berlin}
	%\usetheme{Boadilla}
	%\usetheme{CambridgeUS}
	%\usetheme{Copenhagen}
	%\usetheme{Darmstadt}
	%\usetheme{Dresden}
	%\usetheme{Frankfurt}
	%\usetheme{Goettingen}
	%\usetheme{Hannover}
	%\usetheme{Ilmenau}
	%\usetheme{JuanLesPins}
	%\usetheme{Luebeck}
	\usetheme{Madrid}
	%\usetheme{Malmoe}
	%\usetheme{Marburg}
	%\usetheme{Montpellier}
	%\usetheme{PaloAlto}
	%\usetheme{Pittsburgh}
	%\usetheme{Rochester}
	%\usetheme{Singapore}
	%\usetheme{Szeged}
	%\usetheme{Warsaw}
	
	% As well as themes, the Beamer class has a number of color themes
	% for any slide theme. Uncomment each of these in turn to see how it
	% changes the colors of your current slide theme.
	
	%\usecolortheme{albatross}
	%\usecolortheme{beaver}
	%\usecolortheme{beetle}
	%\usecolortheme{crane}
	%\usecolortheme{dolphin}
	%\usecolortheme{dove}
	%\usecolortheme{fly}
	%\usecolortheme{lily}
	%\usecolortheme{orchid}
	%\usecolortheme{rose}
	%\usecolortheme{seagull}
	%\usecolortheme{seahorse}
	%\usecolortheme{whale}
	%\usecolortheme{wolverine}
	
	%\setbeamertemplate{footline} % To remove the footer line in all slides uncomment this line
	%\setbeamertemplate{footline}[page number] % To replace the footer line in all slides with a simple slide count uncomment this line
	
	%\setbeamertemplate{navigation symbols}{} % To remove the navigation symbols from the bottom of all slides uncomment this line
}

\usepackage{graphicx} % Allows including images
\usepackage[utf8]{vietnam}
\usepackage{booktabs} % Allows the use of \toprule, \midrule and \bottomrule in tables

%----------------------------------------------------------------------------------------
%	TITLE PAGE
%----------------------------------------------------------------------------------------

\title[Short title]{Taxi vận chuyển người kết hợp hàng hóa} % The short title appears at the bottom of every slide, the full title is only on the title page

\author{Trần Huy Hùng,\\ Đỗ Ngọc Sơn} % Your name
\institute[UCLA] % Your institution as it will appear on the bottom of every slide, may be shorthand to save space
{
	Đại học Bách Khoa Hà Nội \\ % Your institution for the title page
	\medskip
	% 	\textit{john@smith.com} % Your email address
}
\date{\today} % Date, can be changed to a custom date

\begin{document}
	
	\begin{frame}
		\titlepage % Print the title page as the first slide
	\end{frame}
	
	\begin{frame}
		\frametitle{Nội dung} % Table of contents slide, comment this block out to remove it
		\tableofcontents % Throughout your presentation, if you choose to use \section{} and\subsection{} commands, these will automatically be printed on this slide as an overview of your presentation
	\end{frame}
	
	%----------------------------------------------------------------------------------------
	%	PRESENTATION SLIDES
	%----------------------------------------------------------------------------------------
	
	%------------------------------------------------
	\section{Giới thiệu bài toán} % Sections can be created in order to organize your presentation into discrete blocks, all sections and subsections are automatically printed in the table of contents as an overview of the talk
	%------------------------------------------------
	
	%\subsection{Subsection Example} % A subsection can be created just before a set of slides with a common theme to further break down your presentation into chunks
	
	\begin{frame}
		\frametitle{Giới thiệu bài toán}
		Lorem ipsum dolor sit amet, consectetur adipiscing elit. Integer lectus nisl, ultricies in feugiat rutrum, porttitor sit amet augue. Aliquam ut tortor mauris. Sed volutpat ante purus, quis accumsan dolor.
	\end{frame}

	%------------------------------------------------
	
	\begin{frame}
		\frametitle{Giới thiệu bài toán}
		Lorem ipsum dolor sit amet, consectetur adipiscing elit. Integer lectus nisl, ultricies in feugiat rutrum, porttitor sit amet augue. Aliquam ut tortor mauris. Sed volutpat ante purus, quis accumsan dolor.
	\end{frame}
	
	%------------------------------------------------
	
	\section{Các hướng tiếp cận}
	
	%------------------------------------------------
	
	\begin{frame}
		\frametitle{Các hướng tiếp cận}
		\begin{itemize}
			\item Sử dụng các giải thuật chính xác
			\item Sử dụng các giải thuật xấp xỉ
		\end{itemize}
	\end{frame}
	
	%------------------------------------------------
	
	\begin{frame}
		\frametitle{Các hướng tiếp cận}
		\begin{block}{Giải bài toán bằng thuật toán chính xác}
			Lorem ipsum dolor sit amet, consectetur adipiscing elit. Integer lectus nisl, ultricies in feugiat rutrum, porttitor sit amet augue. Aliquam ut tortor mauris. Sed volutpat ante purus, quis accumsan dolor.
		\end{block}
	\end{frame}

	%------------------------------------------------
	
	\begin{frame}
		\frametitle{Các hướng tiếp cận}
		\begin{block}{Giải bài toán bằng thuật toán xấp xỉ}
			Lorem ipsum dolor sit amet, consectetur adipiscing elit. Integer lectus nisl, ultricies in feugiat rutrum, porttitor sit amet augue. Aliquam ut tortor mauris. Sed volutpat ante purus, quis accumsan dolor.
		\end{block}
	\end{frame}
	
	%------------------------------------------------
	
	\section{Mô hình hóa bài toán}
	
	%------------------------------------------------
	
	\begin{frame}
		\frametitle{Mô hình ràng buộc}
		\textbf{\underline{Tham số}}
		\begin{itemize}
			\item Tập $2N+2M+K$ điểm:
			\begin{itemize}
				\item Hành khách $i$: điểm đón $i$ và điểm trả $i+N+M$ ($i=1,2,...,N$)
				\item Gói hàng $j$: điểm lấy hàng $j$ và điểm trả $j+N+M$ ($j=N+1,N+2,...,N+M$)
				\item $K$ điểm logic $2N+2M+1, 2N+2M+2, ..., 2N+2M+K$ tham chiếu tới điểm xuất phát vật lý $0$. Điểm $2N+2M+k$ tương ứng là điểm bắt đầu và kết thúc lộ trình xe thứ $k$ ($k=1,2,...,K$)
			\end{itemize}
			\item $d_{ij}$: Khoảng cách từ điểm $i$ tới điểm $j$ ($i,j\in \{1,2,...,2N+2M+K\}$)
			\item $w_i$: Sự thay đổi khối lượng hàng khi đi tới điểm $i$ ($i=1,2,...,2N+2M+K$)
			\begin{equation}
				w_i =
				\begin{cases}
					q_i & \text{nếu } N+1\leq i\leq N+M\\
					-q_{i-(N+M)} & \text{nếu } 2N+M+1\leq i\leq 2N+2M\\
					0 & \text{ngược lại}
				\end{cases} \notag
			\end{equation}
			\item $Q_k$: Khối lượng hàng tối đa xe thứ $k$ có thể chở ($k=1,2,...,K$)
		\end{itemize}
	\end{frame}
	\begin{frame}
		\frametitle{Mô hình ràng buộc}
		\textbf{\underline{Biến quyết định}}
		\begin{itemize}
			\item $x_{ij}$: Biến nhị phân, xác định cung đi từ điểm $i$ đến điểm $j$ có xuất hiện trong lộ trình của 1 trong $k$ xe không ($i,j\in \{1,2,...,2N+2M+K$)
			\begin{equation}
				x_{ij} = 
				\begin{cases}
					1 & \text{nếu cung $(i,j)$ có trong lộ trình của 1 xe}\\
					0 & \text{ngược lại}
				\end{cases}
			\end{equation}
			\item Tại mỗi điểm $i$ ($i=1,2,...,2N+2M+K$):
			\begin{itemize}
				\item $r_i$: chỉ số của xe đi qua điểm $i$ trong lộ trình
				\begin{equation}
					1\leq r_i\leq K
				\end{equation}
				\item $t_i$: thứ tự của điểm $i$ trong lộ trình của xe $k$ đi qua nó (điểm xuất phát có thứ tự 0)
				\begin{equation}
					0\leq t_i\leq 2N+2M
				\end{equation}
				\item $c_i$: khối lượng hàng xe $k$ (đi qua điểm $i$) còn chịu được khi đi tới điểm $i$
				\begin{equation}
					0\leq c_i\leq \max _{1\leq k\leq K} \{Q_k\}
				\end{equation}
			\end{itemize}
		\end{itemize}
	\end{frame}

	\begin{frame}
		\frametitle{Mô hình ràng buộc}
		\textbf{\underline{Các ràng buộc}}
		\begin{itemize}
			\item Ràng buộc cân bằng luồng vào ra:
			\begin{align}
				\sum_{j=1}^{2N+2M+K} x_{ij} = 1,\quad & i=1,2,...,2N+2M+K \\
				\sum_{i=1}^{2N+2M+K} x_{ij} = 1,\quad & j=1,2,...,2N+2M+K
			\end{align}
			\item Xác định $r_i$:
			\begin{align}
				r_{2N+2M+k} = k,\quad & k=1,2,...,K \\
				x_{ij}=1\Rightarrow r_j=r_i,\quad & i=1,2,...,2N+2M+K, \\
				& j=1,2,...,2N+2M, i\ne j \notag
			\end{align}
		\end{itemize}
	\end{frame}
	\begin{frame}
		\frametitle{Mô hình ràng buộc}
		\begin{itemize}
			\item Xác định $t_i$:
			\begin{align}
				t_{2N+2M+k} = 0,\quad & k=1,2,...,K \\
				x_{ij}=1\Rightarrow t_j=t_i + 1,\quad & i=1,2,...,2N+2M+K, \\
				& j=1,2,...,2N+2M,i\ne j \notag
			\end{align}
			\item Xác định $c_i$:
			\begin{align}
				c_{2N+2M+k} = Q_k,\quad & k=1,2,...,K \\
				x_{ij}=1\Rightarrow c_j=c_i - w_j,\quad & i=1,2,...,2N+2M+K, \\
				& j=1,2,...,2N+2M,i\ne j \notag
			\end{align}
		\end{itemize}
	\end{frame}
	\begin{frame}
		\frametitle{Mô hình ràng buộc}
		\begin{itemize}
			\item Điểm đón và trả của hành khách $i$ phải thuộc lộ trình của cùng một xe, tương tự với các gói hàng:
			\begin{align}
				r_i = r_{i+N+M},\quad & i=1,2,...,N+M
			\end{align}
			\item Điểm đón khách phải liền trước điểm trả khách:
			\begin{align}
				x_{i,(i+N+M)} = 1,\quad & i=1,2,...,Nj
			\end{align}
			\item Điểm lấy hàng phải ở trước điểm giao hàng:
			\begin{align}
				t_i < t_{i+N+M},\quad & i=N+1,N+2,...,N+M
			\end{align}
			\item Khối lượng còn lại của xe tại mọi thời điểm không âm (đã thỏa mãn).
		\end{itemize}
	\end{frame}
	\begin{frame}
		\frametitle{Mô hình ràng buộc}
		\textbf{\underline{Hàm mục tiêu}}
		\begin{equation}
			\sum_{i=1}^{2N+2M+K} \sum_{j=1}^{2N+2M+K} d_{ij}\times x_{ij} \leftarrow min
		\end{equation}
	\end{frame}

	\begin{frame}
		\frametitle{Mô hình MIP}
		\textbf{\underline{Tham số}}
		\begin{itemize}
			\item Tập $2N+2M+K$ điểm:
			\begin{itemize}
				\item Hành khách $i$: điểm đón $i$ và điểm trả $i+N+M$ ($i=1,2,...,N$)
				\item Gói hàng $j$: điểm lấy hàng $j$ và điểm trả $j+N+M$ ($j=N+1,N+2,...,N+M$)
				\item $K$ điểm logic $2N+2M+1, 2N+2M+2, ..., 2N+2M+K$ tham chiếu tới điểm xuất phát vật lý $0$. Điểm $2N+2M+k$ tương ứng là điểm bắt đầu và kết thúc lộ trình xe thứ $k$ ($k=1,2,...,K$)
			\end{itemize}
			\item $d_{ij}$: Khoảng cách từ điểm $i$ tới điểm $j$ ($i,j\in \{1,2,...,2N+2M+K\}$)
			\item $w_i$: Sự thay đổi khối lượng hàng khi đi tới điểm $i$ ($i=1,2,...,2N+2M+K$)
			\begin{equation}
				w_i =
				\begin{cases}
					q_i & \text{nếu } N+1\leq i\leq N+M\\
					-q_{i-(N+M)} & \text{nếu } 2N+M+1\leq i\leq 2N+2M\\
					0 & \text{ngược lại}
				\end{cases} \notag
			\end{equation}
			\item $Q_k$: Khối lượng hàng tối đa xe thứ $k$ có thể chở ($k=1,2,...,K$)
		\end{itemize}
	\end{frame}
	\begin{frame}
		\frametitle{Mô hình MIP}
		\textbf{\underline{Biến quyết định}}
		\begin{itemize}
			\item $x_{ij}$: Biến nhị phân, xác định cung đi từ điểm $i$ đến điểm $j$ có xuất hiện trong lộ trình của 1 trong $k$ xe không ($i,j\in \{1,2,...,2N+2M+K$)
			\begin{equation}
				x_{ij} = 
				\begin{cases}
					1 & \text{nếu cung $(i,j)$ có trong lộ trình của 1 xe}\\
					0 & \text{ngược lại}
				\end{cases}
			\end{equation}
			\item Tại mỗi điểm $i$ ($i=1,2,...,2N+2M+K$):
			\begin{itemize}
				\item $r_i$: chỉ số của xe đi qua điểm $i$ trong lộ trình
				\begin{equation}
					1\leq r_i\leq K
				\end{equation}
				\item $t_i$: thứ tự của điểm $i$ trong lộ trình của xe $k$ đi qua nó (điểm xuất phát có thứ tự 0)
				\begin{equation}
					0\leq t_i\leq 2N+2M
				\end{equation}
				\item $c_i$: khối lượng hàng xe $k$ (đi qua điểm $i$) còn chịu được khi đi tới điểm $i$
				\begin{equation}
					0\leq c_i\leq \max _{1\leq k\leq K} \{Q_k\}
				\end{equation}
			\end{itemize}
		\end{itemize}
		
	\end{frame}
	\begin{frame}
		\frametitle{Mô hình MIP}
		\textbf{\underline{Ràng buộc}}
		\begin{itemize}
			\item Ràng buộc cân bằng luồng vào ra:
			\begin{align}
				\sum_{j=1}^{2N+2M+K} x_{ij} = 1,\quad & i=1,2,...,2N+2M+K \\
				\sum_{i=1}^{2N+2M+K} x_{ij} = 1,\quad & j=1,2,...,2N+2M+K
			\end{align}
			\item Xác định $r_i$:
			\begin{align}
				r_{2N+2M+k} = k,\quad & k=1,2,...,K \\
				r_j - r_i\leq \mu\times (1 - x_{ij}),\quad & i=1,2,...,2N+2M+K, \\
				r_j - r_i\geq -\mu\times (1 -x_{ij}),\quad & j=1,2,...,2N+2M, i\ne j \notag
			\end{align}
		\end{itemize}
	\end{frame}
	\begin{frame}
		\frametitle{Mô hình MIP}
		\begin{itemize}
			\item Xác định $t_i$:
			\begin{align}
				t_{2N+2M+k} = 0,\quad & k=1,2,...,K \\
				t_j - t_i -1\leq \mu\times (1 - )x_{ij}),\quad & i=1,2,...,2N+2M+K, \\
				t_j - t_i -1\geq -\mu\times (1 - x_{ij}),\quad & j=1,2,...,2N+2M, i\ne j \notag
			\end{align}
			\item Xác định $c_i$:
			\begin{align}
				c_{2N+2M+k} = Q_k,\quad & k=1,2,...,K \\
				c_j - c_i - w_j\leq \mu\times (1 - x_{ij}),\quad & i=1,2,...,2N+2M+K, \\
				c_j - c_i - w_j\geq -\mu\times (1 - x_{ij}),\quad & j=1,2,...,2N+2M, i\ne j \notag
			\end{align}
		\end{itemize}
	\end{frame}
	\begin{frame}
		\frametitle{Mô hình MIP}
		\begin{itemize}
			\item Điểm đón và trả của hành khách $i$ phải thuộc lộ trình của cùng một xe, tương tự với các gói hàng:
			\begin{align}
				r_i = r_{i+N+M},\quad & i=1,2,...,N+M
			\end{align}
			\item Điểm đón khách phải liền trước điểm trả khách:
			\begin{align}
				x_{i,(i+N+M)} = 1,\quad & i=1,2,...,N
			\end{align}
			\item Điểm lấy hàng phải ở trước điểm giao hàng:
			\begin{align}
				t_i + 1\leq t_{i+N+M},\quad & i=N+1,N+2,...,N+M
			\end{align}
			\item Khối lượng còn lại của xe tại mọi thời điểm không âm (đã thỏa mãn).
		\end{itemize}
	\end{frame}
	\begin{frame}
		\frametitle{Mô hình MIP}
		\textbf{\underline{Ràng buộc thừa}}
		\begin{itemize}
			\item Các điểm logic tham chiếu đến $0$ không nối lẫn nhau:
			\begin{align}
				x_{ij}=0,\quad & i=2N+2M+1,...,2N+2M+K, \\
				& j=2N+2M+1,...,2N+2M+K, i\ne j \notag
			\end{align}
			\item Một điểm không tự nối tới chính nó, trừ $K$ điểm logic tham chiếu tới $0$:
			\begin{align}
				x_{ii}=0,\quad & i=1,2,...,2N+2M
			\end{align}
		\end{itemize}
	\end{frame}
	\begin{frame}
		\frametitle{Mô hình MIP}
		\textbf{\underline{Hàm mục tiêu}}
		\begin{equation}
			\sum_{i=1}^{2N+2M+K} \sum_{j=1}^{2N+2M+K} d_{ij}\times x_{ij} \leftarrow min
		\end{equation}
	\end{frame}
	
	%------------------------------------------------
	
	\section{Cài đặt thuật toán}
	
	%------------------------------------------------
	
	\begin{frame}
		\frametitle{Cài đặt thuật toán}
		Lorem ipsum dolor sit amet, consectetur adipiscing elit. Integer lectus nisl, ultricies in feugiat rutrum, porttitor sit amet augue. Aliquam ut tortor mauris. Sed volutpat ante purus, quis accumsan dolor.
	\end{frame}
	
	%------------------------------------------------
	
	\begin{frame}
		\frametitle{Cài đặt thuật toán}
		Lorem ipsum dolor sit amet, consectetur adipiscing elit. Integer lectus nisl, ultricies in feugiat rutrum, porttitor sit amet augue. Aliquam ut tortor mauris. Sed volutpat ante purus, quis accumsan dolor.
	\end{frame}
	
	%------------------------------------------------
	\section{Thực nghiệm và đánh giá}
	
	%------------------------------------------------
	
	\begin{frame}
		\frametitle{Thực nghiệm và đánh giá}
		Lorem ipsum dolor sit amet, consectetur adipiscing elit. Integer lectus nisl, ultricies in feugiat rutrum, porttitor sit amet augue. Aliquam ut tortor mauris. Sed volutpat ante purus, quis accumsan dolor.
	\end{frame}


	%------------------------------------------------
	
	\begin{frame}
	\frametitle{Thực nghiệm và đánh giá}
		Lorem ipsum dolor sit amet, consectetur adipiscing elit. Integer lectus nisl, ultricies in feugiat rutrum, porttitor sit amet augue. Aliquam ut tortor mauris. Sed volutpat ante purus, quis accumsan dolor.
	\end{frame}

	%------------------------------------------------
	
	\section{Kết luận}
	
	%------------------------------------------------
	
	\begin{frame}
		\frametitle{Kết luận}
		Lorem ipsum dolor sit amet, consectetur adipiscing elit. Integer lectus nisl, ultricies in feugiat rutrum, porttitor sit amet augue. Aliquam ut tortor mauris. Sed volutpat ante purus, quis accumsan dolor.
	\end{frame}
	
\end{document} 